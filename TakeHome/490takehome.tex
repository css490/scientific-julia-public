\documentclass[12pt]{article}

\setlength{\topmargin}{-.75in} \addtolength{\textheight}{2.00in}
\setlength{\oddsidemargin}{.00in} \addtolength{\textwidth}{.75in}

\usepackage{amsmath,color,graphicx,hyperref}

\nofiles

\pagestyle{plain}

\setlength{\parindent}{0in}


\begin{document}

\noindent {\sc {\bf {\Large Take-Home Final}}
            \hfill Elements of Scientific Computing}
\bigskip

%\noindent {\sc  {\large Name:}}
%\bigskip

{\bf Instructions: }Please create a folder in your GitHub repository called ``thfinal'' and for each problem below push a \emph{different} file (as required per the problem) to this folder.\\ 

{\bf Problem 1. }(8 pts) (\emph{Numerical Integration}) In class, we have discussed various methods of numerical integration to solve for the integral of functions of a single variable, however we can easily generalize these techniques to solve for the integrals of multivariate functions. \\

That is, for a surface in 3-dimensional space we can approximate the volume under the surface, and above a particular region on the 2D plane below it, by adding up the volumes of many thin parallelepipeds. For example, if we want to use the midpoint rule with a square base for each of the rectangular parallelepiped subdivisions, then we get:
\[
\int_c^d \int_a^b f(x,y) \; dxdy \approx \Delta x \Delta y \cdot [f(x^*_1,y^*_1) + f(x^*_1,y^*_2) + ... + f(x^*_1,y^*_n) + f(x^*_2,y^*_1) + ... + f(x^*_n,y^*_n)]
\]

where $\Delta x = \frac{b - a}{n}$ and $\Delta y = \frac{d - c}{n}$ when subdividing the region under the surface into a $n^2$ grid. Also, $x^*_i$ here is the midpoint of the interval $[x_{i-1}, x_i]$, while $y^*_j$ is the midpoint of the interval $[y_{j-1}, y_{j}]$. Note: this definition is 0-indexed.\\

Write a program in Julia to evaluate the double integral 
\[
\int_0^1 \int_0^1 e^{-xy} \; dxdy
\]

using the midpoint rule as defined above. Subdivide the $[0,1] \times [0,1]$ region into 10,000 squares. \\ 

\emph{\color{magenta} You should push to GitHub your Julia code for this problem in a .ipynb or .jl file.}\\

{\bf Problem 2. }(8 pts) (\emph{Time-Dependent Heat Equation}) Consider the time dependent heat equation:
\[ 
\frac{\partial u}{\partial t} - \frac{\partial^2 u}{\partial x^2} = 0
\]

for $0 \leq x \leq 1$ and $t \geq 0$, with boundary conditions:
\[
u(t,0) = 0, \; u(t,1) = 0,
\]
and initial condition:
\[
u(0,x) = \begin{cases}
2x & \text{if } 0 \leq x \leq 0.5\\
2 - 2x & \text{if } 0.5 \leq x \leq 1
\end{cases}
\]

Discretize the given heat equation with $\Delta x = 0.05$ and $\Delta t = 0.0012$ (see lecture 3 notes), and solve this problem in Julia in the time interval $[0, 0.06]$. Plot the solution for every $10\Delta t$ starting with the plot for $t=0$ and ending with the plot for $t = 0.06$ using PyPlot. For inspiration on this problem, \href{http://www.colorado.edu/geography/class_homepages/geog_4023_s07/labs/html/PDE_lab.html}{here} is an example on solving the time dependent heat equation using Matlab.\\ 

\emph{\color{magenta} You should push to GitHub your Julia code for this problem in a .ipynb or .jl file.}\\
\newpage

{\bf Problem 3. }(6 pts) (\emph{Logistic Regression and Optimization}) For this problem, consider the \verb+LogReg.ipynb+ notebook from lecture 8, in particular the snippet of code implementing the gradient descent algorithm for logistic regression:
\begin{verbatim}
function gradientDescent(X, y, theta, alpha, iter)
    m = length(y)
    Jhist = zeros(iter,1)
    for k=1:iter
        delta = zeros(size(theta)) 
        for i=1:m
            delta += (1/m)*(sigmoid(X[i,:]*theta) - y[i])[1]*(X[i,:])'
        end
        theta = theta - (alpha/m)*delta
        Jhist[k] = cost(X, y, theta)[1]
    end
    return theta, Jhist
end
\end{verbatim}

Play around with the initial \verb+theta+, \verb+alpha+ and \verb+iter+ inputs and observe \verb+Jhist+. For \emph{each combination} of these inputs (you should try at least 5 combinations!) write two or three sentences describing what is going on with the output. Finally, write a paragraph with a hypothesis as to why the gradient descent algorithm does not perform very well on this logistic regression problem. Note: you do not need to find a combination that causes it to converge.\\

\emph{\color{magenta} You should push to GitHub your .tex and .pdf files for this problem.}\\

{\bf Problem 4. }(8 pts) (\emph{Student Presentations}) Answer the following questions regarding the topics that your fellow students have presented this quarter:
\begin{enumerate}
\item[a) ](2 pts) Say you did something wrong on this take-home and you would like to replace your local changes with the last content pushed to your repository, how can you do that using \verb+git+?
\item[b) ](2 pts) Consider the following snippet of code:
\begin{verbatim}
nprocs()
addprocs(4)
a = dones((4000,1), workers()[1:4], [4,1])
b = {@spawnat p sum(localpart(a)) for p=procs(a)}
c = map(fetch, b)
sum(c)
\end{verbatim}
At first without running the code, what do you think the output will be? Paste it into IJulia, run it and verify your answer. Add in-line comments to each line describing what the code on that line is doing.
\newpage
\item[c) ](2 pts) Multiply two $4 \times 4$ matrices of your choice (non-zero, non-identity matrices only!) using the \emph{Strassen Algorithm}. You may do it by hand and typeset your work in \LaTeX $\;$ or you may implement this using Julia.
\item[d) ](2 pts) Write a function in Julia that takes as input a comma separated file and outputs a tab separated file with the same data contained in the input. \\

\emph{\color{magenta} You should push to GitHub your .tex and .pdf files, as well as your Julia code for all parts of this problem in a .ipynb or .jl file.}\\

\end{enumerate}
\end{document}
